\section{Introduction}\label{sec:introduction}

In today's world, sustainability has become a growing concern as people become more aware of the environmental impact of their daily choices. With this in mind, it is important to consider the environmental implications of often overlooked practices, such as note-taking during university studies.

On average, every year KU Leuven enrolls about 27000 students for bachelor's studies and another 21000 students enroll for master's programs~\cite{KULstatistics}. By far, these two categories represent the majority of the student population, with a high rate of education continuation from bachelor's to master's. This entire process takes on average 5 years, but can easily be extended further. This translates to hundreds of thousands of lecture attendees per year, each requiring (with varying levels of intensity) a way of capturing all the information being taught in an organized fashion. Throughout the years, the overwhelmingly dominant way of achieving this has been through the reliable, tried-and-tested way: pen and paper. However, with technological advancements, more and more students are making the switch towards digital note-taking, raising the question: is this switch a net-positive in our individual environmental impact?

This paper aims to answer this question to the best degree possible. With this focus, the environmental impact of using multiple notebooks and pens throughout a span of 5 years as a student is compared with the impact of using a tablet with a stencil. This is done through a life cycle assessment (LCA) for the multiple components required in both scenarios, which is conducted in the \textsc{SimaPro} software. By shedding light on the environmental impact of note-taking methods, this investigation aims to contribute to the growing body of knowledge on sustainability and to encourage more sustainable practices in education and beyond.