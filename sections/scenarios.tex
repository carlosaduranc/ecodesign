\section{Life Cycle Scenarios}\label{sec:scenarios}
Following the definition for both scenarios to be compared, their life cycles must be described. For this, all materials and processes involved in their material sourcing, processing, manufacturing, assembling, transportation, use and disposal are considered. This was done within \textsc{SimaPro}, with the help of the EcoInvent database. Subsections~\ref{subsec:analog_scenario} and~\ref{subsec:digital_scenario} describe each step for the analog and digital scenarios to be compared, respectively. Moreover, subsection~\ref{subsec:case_studies} describes all of the considered case studies for the simulations.


\subsection{Analog}\label{subsec:analog_scenario}
For the analog case, traditionally used brands for all materials were considered. This choice required some assumptions to be taken in order to ensure sufficient and realistic amounts for the materials used over a 5 year student career. For this, a small survey of 10 master in mechanical engineering students was conducted. Students were asked how many courses they take on average per academic year as well as how many notebooks and pens they require per lecture. This amounted, on average, to 8 lectures per year using 1 notebook per lecture. Additionally, the students agreed upon using three multi-colored pens per semester (6 per academic year). These results amounted to a total of 40 notebooks and 30 pens used over a span of 5 academic years.

Moreover, traditionally bought and widely available brands were chosen for the notebooks and pens. This led to the choice of BIC ball-point pens and spiraled, 100 page A4 Oxford notebooks.

\subsubsection*{Assembly}
Once the material selection was carried out, the assemblies for the notebooks and pens could be created. For this, the EcoInvent database was used within the \textsc{SimaPro} software to take advantage of the highly detailed accounts available for the impacts of a wide range of materials and assemblies. The following describes all components involved in each of the assemblies.

\begin{enumerate}
    \item \textbf{Oxford Notebook (x40)}
    \begin{enumerate}
        \item Wire spiral: For the wire spiral holding the pages and covers together, a low-alloyed steel wire was chosen. This model accounts for the manufacturing of the spirals as well as the sourcing for the steel required to produce all 40 spirals.
        \item Covers (x2/ea): For the covers, white-lined chipboards (WLC) were chosen. Additionally, their calendering, giving them the typical shiny look, was accounted for.
        \item Paper (x100/ea): For the notebook pages, 100 gram per square meter (GSM) A4 paper was taken. This is considered standard in many European paper companies and is widely found in many notebook models. Within the \textsc{SimaPro} software, this was modelled using a wood-containing light weight coated (LWC) paper for which all dimensions and densities were accounted for.
    \end{enumerate}
    \item \textbf{BIC Ball-point Pen (x30)} \cite{BICballpoint}
    \begin{enumerate}
        \item Pen barrel: The outer casing for the pen was modelled using polystyrene. Furthermore, this was subjected to a process of injection moulding to account for the energy required for the shaping of the component.
        \item Pen ink tube and cap: Similar to the pen outer casing, the internal ink tube for the ball-point pen, as well as its calp, were set to be manufactured using injection moulding. However, this was modelled by using polypropylene granulate instead of polystyrene.
        \item Pen ink: This sub-assembly presented some difficulties since there were no pre-defined modules within the EcoInvent database for the specific type of ink used in ball-point pens. However, a model existed for printer toner which resembles the chemical constitution of pen ink. This required the addition solvent (benzyl alcohol) in order to account for the difference in solubility as well as density between the two types of ink. This mixture was done using a 1:1 ratio between the toner and solvent.
        \item Pen tip: The pen tip sub-assembly required two main components. First the ball-point was taken to be made out of pressed and honed brass. Lacking specific modelling for these manufacturing processes, a general metal product manufacturing model was taken. Moreover, the casing for the ball point was again taken to be made from injection moulded polypropylene.
    \end{enumerate}
\end{enumerate}

\subsubsection*{Transportation}
With the goal of accounting for the transportation required for the material to arrive at Belgium, an extensive research on manufacturing plants for both companies (Oxford and BIC) was conducted.

Oxford Notebooks is a daughter company to Hamelin, a French-based company. However, the closest manufacturing plant to Belgium, which is its national provider, is located in the Netherlands outside of Venlo. Therefore, to account for the transportation, a small lorry ($> 16$ ft) was considered. The impact of this transportation is measured by its total tkm (ton kilometer) which amounted to 4.7815 tkm for the transport of all 40 notebooks (27.48 kg) over the driving distance between the manufacturing plant in Venlo and Brussels (174 km).

In the case for the BIC ball-point pens, it was found that the provider for Belgium was a manufacturing plant in Montevráin, France. Similar to the notebook case, a small lorry was considered with a total of 0.0474 tkm for all 30 pens over the 482 km between the manufacturing plant and Brussels.

\subsubsection*{Use Case}
Considering that this scenario only involved analog components, the use case of it required to additional contributions to its environmental impact. This presented a clear advantage to the digital scenario, which will be discussed in detail in the following section.

\subsubsection*{Waste Scenario}
For the end of life treatment of all components used over the 5 years of study, a disposal scenario was developed. The developed disposal scenario was based on detailed accounts from both Oxford and BIC for their corresponding disposal programs. However, since these programs require of a proper disposal by the end users, it was decided that only 15\% of the used materials would find their way to them while the remaining 85\% would end up in landfills.

The finalized disposal program consists of the following:
\begin{itemize}
    \item 80\% of the polystyrene from the pens is able to be recycled for further use within BIC manufacturing processes.
    \item In contrast, the polypropylene from the ball-point pens is not able to be process, leading to most of its mass to be exported to developing countries and end up in massive landfills~\cite{BICballpoint}.
    \item 100\% of the brass recovered from the ball-point is recycled.
    \item 100\% of the returned steel from the notebook spirals is made into scrap for further reuse.
    \item 100\% of the papers as well as the covers is able to be turned into recycled paper for further use within Oxford notebooks.
    \item All remaining materials such as leftover ink in the pen cartridges are disposed of into inert material landfills.
\end{itemize}

\subsection{Digital}\label{subsec:digital_scenario}
The selected tablet for this study is the 10th generation of the Apple iPad along with its compatible 1st generation Apple Pencil. The reason why an iPad was selected is because it is the leading product for tablets and note-taking electronic devices. The 10th generation is the last generation of the regular iPad, which is also the most affordable version. Although a 2nd generation of the Apple Pencil already exists, it is not compatible with the 10th generation of the iPad, hence the reason why the 1st generation of the Apple Pencil was considered. Although the materials for both the iPad and Apple Pencil are not publicly disclosed, this materials can be approximated considering the total weight of the devices and their functionality. Apple releases environmental reports for their products, where it describes that the aluminum for the iPad comes from recycled material. The other components such as the wiring board, screen and batteries are strictly confidential for the Apple products, therefore, general components for electronic devices were considered in \textsc{SimaPro}. It is important to mention that the transportation for the raw materials is already accounted for in the EcoInvent database from \textsc{SimaPro}.
\subsubsection*{Assembly}
\begin{enumerate}
    \item \textbf{Apple iPad 10th Generation}
    \begin{enumerate}
        \item Aluminum casing: To configure this component, a new scrap of aluminum was selected. New scrap refers to aluminum that has been recycled once. Apple specifies in its environmental reports that the housing for their iPad models is made from recycled materials and has a weight of 147g. In \textsc{SimaPro}, the aluminum selected includes manufacturing processes such as, melting, alloying and casting. 
        \item PWB: The configuration of the printed wiring board (PWB) was inspired by the one in the 'Mobile phone tutorial' example elaborated in class, however, it was re-dimensioned according to the specifications for the iPad. The selected PWB in \textsc{SimaPro} describes an unspecified, through-hole mounted PWB used in electronic and electric devices. The overall weight of the PWB is 156g. 
        \item LCD screen: A liquid-crystal display (LCD) screen contains a mirror, glass filter, polarizing screen, liquid-crystal layer, among other components. The selected LCD screen in \textsc{SimaPro} accounts for all the parts that compose an LCD module. The data represents a 15-inch LCD screen, however, a new dimension was specified in the configuration according to the dimensions of the iPad. The LCD screen is the lightest component in the assembly, with a total weight of 15g. 
        \item LiIo battery: Apple uses lithium ion (LiIo) batteries for their products and is the heaviest component from the assembly with 159g. The selected battery in \textsc{SimaPro} describes a rechargeable lithium ion battery pack that includes 14 single cells, a steel casing, a battery management system and the required cables. 
    \end{enumerate}
    \item \textbf{Apple Pencil 1st Generation}
    \begin{enumerate}
        \item Plastic casing: The casing for the Apple Pencil is made from a lightweight high-gloss plastic. A typical example for a high-gloss plastic is polycarbonate (PC). It is very common to see PC as a casing material in electronic devices, therefore, this material was selected in \textsc{SimaPro} and configured with a total weight of 42g. 
        \item PWB: The selected printed wiring board in \textsc{SimaPro} is also the one used in the iPad, which describes a PWB used in general electronic devices. The overall weight for this PWB is 5.2g. 
        \item LiIo battery: The configuration of the lithium ion battery for the Apple Pencil is similar to the one described for the iPad. The only difference is the weight of the battery, which for the Apple Pencil is 10.3g. 
        \item Pencil tip: The tip of the Apple Pencil uses a different plastic than the casing. Polypropylene (PP) as granulates was selected for the tips. Although the material for the tip is not publicly disclosed, it is known that it has to be a lightweight plastic where carbon nano tubes have to be embedded to allow conductivity, and PP is usually used for this application. This is the lightest component of the assembly, with an overall weight of 1g.  
    \end{enumerate}
\end{enumerate}

\subsubsection{Transportation}
To account for the transportation for the iPad and Apple Pencil, it is crucial to consider the manufacturing plants for these products. Apple manufactures its products in a megafactory located in Shenzhen, China. It was considered that this products will travel from Shenzhen to Rotterdam by transoceanic freight ship, with a total impact of 5.6275 tkm over a total distance of 18,256 km. Then the products are transported by lorry ($> 16$ ft) from Rotterdam to a retail store in Brussels, with an impact of 0.1045 tkm with a total distance of 149.8 km. 

\subsubsection*{Use Case}
In contrast to the analog alternative previously described, the energy required to use the digital components has to be accounted for. The energy required for the iPad and Apple Pencil was taken as whole in \textsc{SimaPro}, since the Apple Pencil is charged with the iPad. The selected scenario in \textsc{SimaPro} is inspired from the use of a laptop, which takes into account an active use of 6.5 hours a day for 240 days a year. 

\subsubsection*{Waste Scenario}
The life cycle for the digital alternative also considers a 5-year university study. To assess the waste scenario for the iPad and Apple Pencil, 15\% was considered to be properly disposed by the user and 85\% will go to landfill. The proper disposal is detailed as follows:

\begin{enumerate}
    \item 100\% of the aluminum used for the housing of the iPad can be recycled. 
    \item 100\% of the lithium ion batteries can be properly disposed. 
    \item 100\% of the printed wiring boards are properly disposed for precious metal recovery. 
    \item 100\% of the polypropylene used for the Apple Pencil tips is correctly discarded. 
    \item 100\% of the waste plastic for consumer electronics, such as polycarbonate is properly disposed. 
    \item 100\% of the LCD screen for the iPad is dismantled mechanically, which includes the manual disassembly of the LCD and shredding of the remaining part. 
\end{enumerate}

\subsection{Case Studies}\label{subsec:case_studies}
In order to allow multiple perspectives for the comparison between both note-taking approaches, multiple scenarios were considered. In the following, the considerations for all scenarios are described in great detail.

\subsubsection*{Manufacturing only}
The first case study considers the environmental impacts of manufacturing an iPad and Apple pencil compared to the materials required for a 5 year academic career. This study does not consider the energy required to charge an iPad. This is done with the goal of decoupling the environmental impact of the use of the iPad from its manufacturing since its impact is fully dependent on the energy source shares offered by the grid, as well as any considerations for local renewable installations such as solar PV roofs. Multiple scenarios with varying renewable energy sources (RES) penetration levels are considered in the remaining case studies. 

To manufacture an iPad, the aluminum body, circuit board, and battery must be produced and assembled. This involves the use of considerable amounts of water, chemicals, and energy, which contribute to air, water and soil pollution. For the production of an Apple pencil, the process is similar, except for the body of the pencil, which is made from injection moulding of plastic. 

In order to manufacture a paper notebook, wood pulp is used to create the paper. It is well known that the production paper also needs large mounts of water and energy. However, chemicals such as dyes and bleaches are also encountered in the manufacturing process. The production of a pen also involves the use of chemical substances such as the ink and solvent, and to manufacture the barrel, plastic injection moulding is used. 

\subsubsection*{5 years: 0\% RES penetration}

Once the manufacturing processes are compared independently from their use cases, the energy required to run the digital components is factored in, for which the impact caused by the digital case is expected to increase significantly. Within the \textsc{SimaPro} software, this was set up by introducing a parameter which represents the balance between energy from the national Belgian grid and a local solar PV installation.

In the first use case scenario, this parameter is set to 0\%, indicating that the energy is entirely sourced from the EcoInvent database for the Belgian grid mix. This scenario represents the situation where the difference in environmental impact between the digital and analog scenarios is expected to be the most significant.

\subsubsection*{5 years: 50\% RES penetration}
Following the case with 0\% RES penetration, a case study with a grid mix containing 50\% renewables is considered. This translates to a local solar PV installation large enough to deliver half of the required energy to charge the components in the digital case over five years.

\subsubsection*{5 years: 100\% RES penetration}
Lastly, a case study is conducted to examine a scenario with 100\% solar PV penetration. This particular case represents an idealistic situation where non-renewable energy sources are completely eliminated from the energy mix, and the electronics are exclusively charged at outlets powered by solar PV energy.