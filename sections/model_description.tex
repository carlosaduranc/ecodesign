\section{Model Description}\label{sec:model_description}
The following section contains the description for all models required for the comparison done in this investigation. First, the functional unit used as a comparison point will be discussed in subsection~\ref{subsec:functional_unit}. After, thorough descriptions for the required models for both scenarios are discussed in subsections~\ref{subsec:inventory}.


\subsection{Functional Unit}\label{subsec:functional_unit}
As a first step, the topic of the functional unit must be discussed. This is a crucial step in the LCA for any product, since its very definition can sway results in a particular way. Therefore, this process was conducted with a large attention to detail in order to provide a neutral comparison between both note-taking approaches. Questions such as \textit{what?}, \textit{how much?}, \textit{how well?}, \textit{where?} and \textit{for how long?} are of special interest, leading to the following functional unit.

\renewcommand{\arraystretch}{1.5}
\begin{table}[H]
    \centering
    \begin{tabular}{M{0.4\textwidth}|M{0.5\textwidth}}
        Question & Function \\
        \hline
        \hline
        \textit{What} should the note-taking approach do?            &   Serve for note-taking for university studies with multi-color abilities\\
        \hline
        \textit{How many} materials should the approach use?         &   Enough to provide enough note-taking space for all courses\\
        \hline
        \textit{How well} should the notes be taken?                 &   Approach should deliver, readable, organized notes\\
        \hline
        \textit{Where} are the notes taken?                          &   At KU Leuven, campus Arenberg\\
        \hline
        \textit{For how long} should the approach last?              &   5 years (bachelor's + master's)
    \end{tabular}
    \caption{Functional unit for comparison between note-taking scenarios.}
    \label{tab:functional_unit}
\end{table}
\renewcommand{\arraystretch}{1}

\subsection{Scenario Inventory}\label{subsec:inventory}
Before discussing the effects for each scenario, an inventory for their unit processes must be done. This inventory contains the raw materials involved in the manufacturing process for all components used in the scenario, all processes required for the assembly as well as the transportation involved in the delivery of all physical products. It is worth noting that the materials listed in the following are not an extensive list since the use of the \textsc{SimaPro} software accounts for many often overlooked materials.

\subsubsection{Model: Pen and Paper}\label{subsubsec:model_pen_paper}

\begin{figure}[H]
    \centering
\begin{tikzpicture}[every text node part/.style={align=center}]
    \node(raw_mat1)[unit]{\footnotesize{Bark, Fibers,}\\\footnotesize{Cotton, etc.}};
    \node(raw_mat2)[unit, right of=raw_mat1, xshift=0.11\textwidth]{\footnotesize{Aluminum}};
    \node(raw_mat3)[unit, right of=raw_mat2, xshift=0.11\textwidth]{\footnotesize{Polystyrene}};
    \node(raw_mat4)[unit, right of=raw_mat3, xshift=0.11\textwidth]{\footnotesize{Polypropylene}};
    \node(raw_mat5)[unit, right of=raw_mat4, xshift=0.11\textwidth]{\footnotesize{Ink, solvents}};
    \node(dots1)[blank, right of=raw_mat5, xshift=0.05\textwidth]{\footnotesize{\dots}};
    
    \node(process1)[unit, below of=raw_mat1, yshift=-0.4cm]{\footnotesize{Pulping, pressing,}\\\footnotesize{calendering}};
    \node(process2)[unit, below of=raw_mat2, yshift=-0.4cm]{\footnotesize{Wire drawing}};
    \node(process3)[unit, below of=raw_mat3, yshift=-0.4cm]{\footnotesize{Injection}\\\footnotesize{moulding}};
    \node(process4)[unit, below of=raw_mat4, yshift=-0.4cm]{\footnotesize{Injection}\\\footnotesize{moulding}};
    \node(process5)[unit, below of=raw_mat5, yshift=-0.4cm]{\footnotesize{Chemical mixing}};
    \node(dots2)[blank, below of=dots1, yshift=-0.4cm]{\footnotesize{\dots}};

    \node(blank1)[blank, below of=process1, yshift=-0.4cm]{};
    \node(blank2)[blank, below of=process2, yshift=-0.4cm]{};
    \node(blank3)[blank, below of=process3, yshift=-0.4cm]{};
    \node(blank4)[blank, below of=process4, yshift=-0.4cm]{};
    \node(blank5)[blank, below of=process5, yshift=-0.4cm]{};
    \node(blank_dots)[blank, below of=dots2, yshift=-0.4cm]{};

    \node(blank_assembly)[blank, below of=blank1, yshift=-0.4cm]{};
    \node(assembly_transport)[unit, right of=blank_assembly, xshift=0.05\textwidth]{\small{Assembly}\\ \small{and}\\\small{Transportation}};
    \node(product)[unit, right of=assembly_transport, xshift=0.2\textwidth]{\small{5 Years University Studies:}\\\small{Oxford Notebooks}\\\small{BIC Ball-point pens}};
    \node(waste)[unit, below of=product, yshift=-1cm]{\small{Recycling}\\\small{Disassembly}\\\small{Landfill}};

    \draw[thick, ->, >=latex](raw_mat1.south) -- (process1.north);
    \draw[thick, ->, >=latex](raw_mat2.south) -- (process2.north);
    \draw[thick, ->, >=latex](raw_mat3.south) -- (process3.north);
    \draw[thick, ->, >=latex](raw_mat4.south) -- (process4.north);
    \draw[thick, ->, >=latex](raw_mat5.south) -- (process5.north);

    \draw[thick](process1.south) -- (blank_assembly.center);
    \draw[thick](process2.south) -- (blank2.center);
    \draw[thick](process3.south) -- (blank3.center);
    \draw[thick](process4.south) -- (blank4.center);
    \draw[thick](process5.south) -- (blank5.center);
    \draw[thick](blank_dots.center) -- (blank1.center);

    \draw[thick, ->, >=latex](blank_assembly.center) -- (assembly_transport.west);

    \draw[thick, ->, >=latex](assembly_transport.east) -- (product.west);
    \draw[thick, ->, >=latex](product.south) -- (waste.north);
    

\end{tikzpicture}
\caption{Inventory for analog note-taking scenario.}\label{fig:inventory_pen_paper}
\end{figure}


\subsubsection{Model: Tablet and Stylus}\label{subsubsec:model_tablet_stylus}

\begin{figure}[H]
    \centering
\begin{tikzpicture}[every text node part/.style={align=center}]
    \node(raw_mat1)[unit]{\footnotesize{Aluminum}};
    \node(raw_mat2)[unit, right of=raw_mat1, xshift=0.11\textwidth]{\footnotesize{Copper}};
    \node(raw_mat3)[unit, right of=raw_mat2, xshift=0.11\textwidth]{\footnotesize{Tin}};
    \node(raw_mat4)[unit, right of=raw_mat3, xshift=0.11\textwidth]{\footnotesize{Plastic}};
    \node(raw_mat5)[unit, right of=raw_mat4, xshift=0.11\textwidth]{\footnotesize{Gold}};
    \node(dots1)[blank, right of=raw_mat5, xshift=0.05\textwidth]
    {\footnotesize{Rare earth elements}};
    \node(dots1)[blank, right of=raw_mat5, xshift=0.05\textwidth]
    {\footnotesize{\dots}};
    
    \node(process1)[unit, below of=raw_mat1, yshift=-0.4cm]{\footnotesize{Process 1}};
    \node(process2)[unit, below of=raw_mat2, yshift=-0.4cm]{\footnotesize{Process 2}};
    \node(process3)[unit, below of=raw_mat3, yshift=-0.4cm]{\footnotesize{Process 3}};
    \node(process4)[unit, below of=raw_mat4, yshift=-0.4cm]{\footnotesize{Process 4}};
    \node(process5)[unit, below of=raw_mat5, yshift=-0.4cm]{\footnotesize{Process 5}};
    \node(dots2)[blank, below of=dots1, yshift=-0.4cm]{\footnotesize{\dots}};

    \node(blank1)[blank, below of=process1, yshift=-0.4cm]{};
    \node(blank2)[blank, below of=process2, yshift=-0.4cm]{};
    \node(blank3)[blank, below of=process3, yshift=-0.4cm]{};
    \node(blank4)[blank, below of=process4, yshift=-0.4cm]{};
    \node(blank5)[blank, below of=process5, yshift=-0.4cm]{};
    \node(blank_dots)[blank, below of=dots2, yshift=-0.4cm]{};

    \node(blank_assembly)[blank, below of=blank1, yshift=-0.4cm]{};
    \node(assembly_transport)[unit, right of=blank_assembly, xshift=0.05\textwidth]{\small{Assembly}\\ \small{and}\\\small{Transportation}};
    \node(product)[unit, right of=assembly_transport, xshift=0.2\textwidth]{\small{5 Years University Studies:}\\\small{Apple iPad 10th Gen}\\\small{Apple Pencil 1st Gen}};
    \node(electricity)[unit, right of=product, xshift=0.2\textwidth]{\small{Electricity}};
    \node(waste)[unit, below of=product, yshift=-1cm]{Waste};

    \draw[thick, ->, >=latex](raw_mat1.south) -- (process1.north);
    \draw[thick, ->, >=latex](raw_mat2.south) -- (process2.north);
    \draw[thick, ->, >=latex](raw_mat3.south) -- (process3.north);
    \draw[thick, ->, >=latex](raw_mat4.south) -- (process4.north);
    \draw[thick, ->, >=latex](raw_mat5.south) -- (process5.north);

    \draw[thick](process1.south) -- (blank_assembly.center);
    \draw[thick](process2.south) -- (blank2.center);
    \draw[thick](process3.south) -- (blank3.center);
    \draw[thick](process4.south) -- (blank4.center);
    \draw[thick](process5.south) -- (blank5.center);
    \draw[thick](blank_dots.center) -- (blank1.center);

    \draw[thick, ->, >=latex](blank_assembly.center) -- (assembly_transport.west);

    \draw[thick, ->, >=latex](assembly_transport.east) -- (product.west);
    \draw[thick, ->, >=latex](electricity.west) -- (product.east);
    \draw[thick, ->, >=latex](product.south) -- (waste.north);
    

\end{tikzpicture}
\caption{Caption.}\label{fig:inventory_ipad}
\end{figure}