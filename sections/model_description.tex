\section{Model Description}\label{sec:model_description}
The following section contains the description for all models required for the comparison done in this investigation. First, the functional unit used as a comparison point will be discussed in subsection~\ref{subsec:functional_unit}. After, thorough descriptions for the required models for both scenarios are discussed in subsections~\ref{subsec:model_pen_paper} and~\ref{subsec:model_tablet_stylus}.


\subsection{Functional Unit}\label{subsec:functional_unit}
As a first step, the topic of the functional unit must be discussed. This is a crucial step in the LCA for any product, since its very definition can sway results in a particular way. Therefore, this process was conducted with a large attention to detail in order to provide a neutral comparison between both note-taking approaches. Questions such as \textit{what?}, \textit{how much?}, \textit{how well?}, \textit{where?} and \textit{for how long?} are of special interest, leading to the following functional unit.

\renewcommand{\arraystretch}{1.5}
\begin{table}[H]
    \centering
    \begin{tabular}{M{0.4\textwidth}|M{0.5\textwidth}}
        Question & Function \\
        \hline
        \hline
        \textit{What} should the note-taking approach do?            &   Serve for note-taking for university studies with multi-color abilities\\
        \hline
        \textit{How many} materials should the approach use?         &   Enough to provide enough note-taking space for all courses\\
        \hline
        \textit{How well} should the notes be taken?                 &   Approach should deliver, readable, organized notes\\
        \hline
        \textit{Where} are the notes taken?                          &   At KU Leuven, campus Arenberg\\
        \hline
        \textit{For how long} should the approach last?              &   5 years (bachelor's + master's)
    \end{tabular}
    \caption{Functional unit for comparison between note-taking scenarios.}
    \label{tab:functional_unit}
\end{table}
\renewcommand{\arraystretch}{1}

\subsection{Model: Pen and Paper}\label{subsec:model_pen_paper}


\subsection{Model: Tablet and Stylus}\label{subsec:model_tablet_stylus}