\section{Conclusions}\label{sec:conclusions}

Two different note-taking approaches were considered for the LCA of a 5-year university study period presented in this report. First, an \textit{analog} alternative is defined, which consists on using a number of Oxford paper notebooks and BIC ball-point pens. Furthermore, a \textit{digital} alternative that describes the use of an iPad and Apple Pencil for note taking is defined. The models were thoroughly described, giving the functional unit and their inventory. The latter consists on the materials and manufacturing processes used for their fabrication, while the former answers questions such as \textit{what?, where?, how many?, how well?, for how long?}. 

The life cycle scenarios for each note-taking alternative were also defined, which consisted in the assemblies, use case, transportation and waste scenarios. For the Oxford notebooks, it was determined that he closest manufacturing plant is located in the Netherlands, while the BIC pens are manufactured in France. The Apple products come all the way from its megafactory in China, which travel by freight ship to Rotterdam. It was determined that for both cases, 15\% of the materials find a proper end-of-life, while the remaining 85\% ends up in landfill. 

The case studies for the LCA considered only the manufacturing of the products, a 0\% RES penetration, 50\% RES penetration, and 100\% RES penetration. The results were presented for each of the scenarios previously mentioned. It was determined that for the manufacturing scenario, the digital alternative has less overall impact than the analog option. Nevertheless, this analysis did not consider the energy required to charge the Apple products. For the remaining three scenarios, the analog products showed less overall impact than the digital products, regardless on how much energy comes from renewable sources. However, at its largest point (100\% renewable) the total single score for both scenarios only differ in 3 eco-points. Considering that the electronics used in the digital scenario can also be used for different activities outside university work, it can be said that this scenario represents a viable option as an alternative for paper note-taking.

When comparing all RES penetration scenarios, a trend for an increasing single score was established. Namely, this was experienced in the human health category, showcasing slight increases between all three scenarios. By comparing their midpoint assessments, it was seen that, for all categories affecting human health, only an increase in the impact on human health due to water consumption was present. For all remaining categories, the impact on human health decreased as the RES penetration increased. The increase for the single score can be attributed to the weight factors given to the water consumption, as it is the main driver for their single score.