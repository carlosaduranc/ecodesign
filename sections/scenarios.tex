\section{Life Cycle Scenarios}\label{sec:scenarios}
Following the definition for both scenarios to be compared, their life cycles must be described. For this, all materials and processes involved in their material sourcing, processing, manufacturing, assembling, transportation, use and disposal are considered. This was done within \textsc{SimaPro}, with the help of the EcoInvent database. Subsections~\ref{subsec:analog_scenario} and~\ref{subsec:digital_scenario} describe each step for the analog and digital scenarios to be compared, respectively. Moreover, subsection~\ref{subsec:case_studies} describes all of the considered case studies for the simulations.


\subsection{Analog}\label{subsec:analog_scenario}

\subsubsection*{Assembly}
\begin{enumerate}
    \item \textbf{Oxford Notebook (x40)}
    \begin{enumerate}
        \item Wire spiral:
        \item Covers (x2/ea):
        \item Paper (x100/ea):
    \end{enumerate}
    \item \textbf{BIC Ball-point Pen (x30)}
    \begin{enumerate}
        \item Pen casing:
        \item Pen ink tube:
        \item Pen tip:
    \end{enumerate}
\end{enumerate}

\subsubsection*{Use Case}

\subsubsection*{Waste Scenario}


\subsection{Digital}\label{subsec:digital_scenario}
The selected tablet for this study is the 10th generation of the Apple iPad along with its compatible 1st generation Apple Pencil. The reason why an iPad was selected is because it is the leading product for tablets and note-taking electronic devices. The 10th generation is the last generation of the regular iPad, which is also the most affordable version. Although a 2nd generation of the Apple Pencil already exists, it is not compatible with the 10th generation of the iPad, hence the reason why the 1st generation of the Apple Pencil was considered. Although the materials for both the iPad and Apple Pencil are not publicly disclosed, this materials can be approximated considering the total weight of the devices and their functionality. Apple releases environmental reports for their products, where it describes that the aluminum for the iPad comes from recycled material. The other components such as the wiring board, screen and batteries are strictly confidential for the Apple products, therefore, general components for electronic devices were considered in Simapro. It is important to mention that the transportation for the raw materials is already accounted for in the Ecoinvent database from Simapro.
\subsubsection*{Assembly}
\begin{enumerate}
    \item \textbf{Apple iPad 10th Generation}
    \begin{enumerate}
        \item Aluminum casing: To configure this component, a new scrap of alumnimum was selected. New scrap refers to aluminum that has been recycled once. Apple specifies in its environmental reports that the housing for their iPad models is made from recycled materials and has a weight of 147g. In Simapro, the aluminum selected includes manufacturing processes such as, melting, alloying and casting. 
        \item PWB: The configuration of the printed wiring board (PWB) was inspired by the one in the 'Mobile phone tutorial' example elaborated in class, however, it was redimensioned according to the specifications for the iPad. The selected PWB in Simapro describes an unspecified, through-hole mounted PWB used in electronic and electric devices. The overall weight of the PWB is 156g. 
        \item LCD screen: A liquid-crystal display (LCD) screen contains a mirror, glass filter, polarizing screen, liquid-crystal layer, among other components. The selected LCD screen in Simapro accounts for all the parts that compose an LCD module. The data represents a 15-inch LCD screen, however, a new dimension was specified in the configuration according to the dimensions of the iPad. The LCD screen is the lightest component in the assembly, with a total weight of 15g. 
        \item LiIo battery: Apple uses lithium ion (LiIo) batteries for their products and is the heaviest component from the assembly with 159g. The selected battery in Simapro describes a rechargeable lithium ion battery pack that includes 14 single cells, a steel casing, a battery management system and the required cables. 
    \end{enumerate}
    \item \textbf{Apple Pencil 1st Generation}
    \begin{enumerate}
        \item Plastic casing: The casing for the Apple Pencil is made from a lightweight high-gloss plastic. A typical example for a high-gloss plastic is polycarbonate (PC). It is very common to see PC as a casing material in electronic devices, therefore, this material was selected in Simapro and configured with a total weight of 42g. 
        \item PWB: The selected printed wiring board in Simapro is also the one used in the iPad, which describes a PWB used in general electronic devices. The overall weight for this PWB is 5.2g. 
        \item LiIo battery: The configuration of the lithium ion battery for the Apple Pencil is similar to the one described for the iPad. The only difference is the weight of the battery, which for the Apple Pencil is 10.3g. 
        \item Pencil tip: The tip of the Apple Pencil uses a different plastic than the casing. Polypropylene (PP) as granulates was selected for the tips. Although the material for the tip is not publicly disclosed, it is known that it has to be a lightweight plastic where carbon nano tubes have to be embedded to allow conductivity, and PP is usually used for this application. This is the lightest component of the assembly, with an overall weight of 1g.  
    \end{enumerate}
\end{enumerate}

\subsubsection*{Use Case}
In contrast to the analog alternative previously described, the energy required to use the digital components has to be accounted for. The energy required for the iPad and Apple Pencil was taken as whole in Simapro, since the Apple Pencil is charged with the iPad. The selected scenario in Simapro is inspired from the use of a laptop, which takes into account an active use of 6.5 hours a day for 240 days a year. 

\subsubsection*{Waste Scenario}
The life cycle for the digital alternative also considers a 5-year university study. To assess the waste scenario for the iPad and Apple Pencil, 15\% was considered to be properly disposed by the user and 85\% will go to landfill. The propal disposal is detailed as follows:

\begin{enumerate}
    \item 100\% of the aluminum used for the housing of the iPad can be recycled. 
    \item 100\% of the lithium ion batteries can be properly disposed. 
    \item 100\% of the printed wiring boards are properly disposed for precious metal recovery. 
    \item 100\% of the polypropylene used for the Apple Pencil tips is correctly discarded. 
    \item 100\% of the waste plastic for consumer electronics, such as polycarbonate is properly disposed. 
    \item 100\% of the LCD screen for the iPad is dismantled mechanically, which includes the manual disassembly of the LCD and shredding of the remaining part. 
\end{enumerate}

\subsection{Case Studies}\label{subsec:case_studies}

\subsubsection*{Manufacturing}
The first case study considers the environmental impacts of manufacturing an iPad and Apple pencil compared to a paper notebook and a pen. It is important to mention that this study does not take into account the energy needed to charge an iPad. Moreover in the following case studies, different energy scenarios are considered and further discussed. 

To manufacture an iPad, the aluminum body, circuit board, and battery must be produced and assembled. This involves the use of considerable amounts of water, chemicals, and energy, which contribute to air, water and soil pollution. For the production of an Apple pencil, the process is similar, except for the body of the pencil, which is made from injection moulding of plastic. 

In order to manufacture a paper notebook, wood pulp is used to create the paper. It is well known that the production paper also needs large mounts of water and energy, however, chemicals such as dyes and bleaches are also encountered in the manufacturing process. The production of a pen also involves the use of chemical substances such as the ink and solvent, and to manufacture the barrel, plastic injection moulding is used. 

[PUT RESULTS AND DISCUSS THEM]

\subsubsection*{5 years: 0\% RES Penetration}

\subsubsection*{5 years: 50\% RES Penetration}

\subsubsection*{5 years: 100\% RES Penetration}
